
\chapter{Troubleshooting}

The simulation in PICSimLab consists of 3 parts:

\begin{itemize}
 \item The microcontroller program
 \item Microcontroller simulation (made by \href{https://github.com/lcgamboa/picsim}{picsim} and \href{https://github.com/buserror/simavr}{simavr})
 \item Simulation of boards and parts
\end{itemize}


When a problem occurs it is important to detect where it is occurring.

One of the most common problems is the error in the microcontroller program. Before creating an issue, test your code on a real circuit (even partially) to make sure the problem is not there.

Errors in the microcontroller simulation can be detected using code debugging. Any instruction execution or peripheral behavior outside the expected should be reported in the project of simulator used ([picsim](https://github.com/lcgamboa/picsim) or [simavr](https://github.com/buserror/simavr)).

If the problem is not in either of the previous two options, the problem is probably in PICSimLab. A good practice is to send a source code together with a PICSimLab workspace (.pzw file) to open the issue about the problem.

