\chapter{Debug Support}

The type of debug interface depends on the backend simulator utilized.


\section{MPLABX Integrated Debug (picsim and simavr) } \hypertarget{def:mplabxd}{}

To use the \href{http://www.microchip.com/mplabx}{MPLABX} IDE for debug and program the PicsimLab, install the plugin \href{https://github.com/lcgamboa/picsimlab_md/releases/}{com-picsim-picsimlab.nbm} in MPLABX.

The plugin connect to Picsimlab through a TCP socket using port 1234 (or other defined in configuration window), and you have to allow the access in the firewall.

\href{https://lcgamboa.github.io/picsimlab_docs/stable/UsewithMPLABX.html}{Tutorial: how to use MPLABX to program and debug PICsimLab}.

It's possible import and debug a Arduino sketch into MPLABX using the \href{https://github.com/janegilruud/chipKIT-importer-2.0}{Arduino import plugin}.

\section{Arduino IDE Integration (simavr) } \hypertarget{def:arduinoide}{}

For integrated use with the Arduino IDE, simply configure the serial port as explained in the Chapter \hyperlink{def:seriali}{Serial Communication} and load the Arduino bootloader. The bootloader can be loaded from the ``Tools->Arduino bootloader'' menu.

In Windows, considering com0com making a NULL-MODEM connection between COM1 and COM2, simply connect the PICSimLab on the COM1 port (defined in configuration window) and the Arduino IDE on the COM2 port or vice versa.

On Linux the operation is the same, but using for example the ports /dev/tnt2 and /dev/tnt3.

In Linux for the virtual ports to be detected in Arduino it is necessary to replace the library lib/liblistSerialsj.so of the Arduino with a version which support the detection of tty0tty ports, that can be downloaded in the link \href{https://github.com/lcgamboa/listSerialPortsC/releases} {listSerialC with tty0tty support}.


\section{avr-gdb Debug (simavr)} \hypertarget{def:gdbavr}{}
 
 With debug support enabled you can use avr-gdb to debug the code used in the simulator. 
 Use the configuration window to choose between MDB (MPLABX) or GDB to debug AVR microcontrollers. 
 
 
 Use avr-gdb with the .elf file as the parameter:
 \begin{verbatim}
 avr-gdb compiled_file.elf
 \end{verbatim}
 and the command below to connect (1234 is the default port):
 \begin{verbatim}
 target remote localhost:1234
 \end{verbatim}

Graphic debug mode can be made using \href{https://www.eclipse.org/}{eclipse IDE} with \href{https://eclipse.baeyens.it/}{Sloeber Arduino plugin}.
 

 \section{arm-gdb Debug (qemu-stm32)} \hypertarget{def:gdbarm}{}
 
 With debug support enabled you can use arm-none-eabi-gdb (or gdb-multiarch) to debug the code used in the simulator. 
 
 Use arm-none-eabi-gdb with the .elf file as the parameter:
 \begin{verbatim}
 arm-none-eabi-gdb compiled_file.elf
 \end{verbatim}
 and the command below to connect (1234 is the default port):
 \begin{verbatim}
 target remote localhost:1234
 \end{verbatim}

Graphic debug mode can be made using \href{https://www.eclipse.org/}{eclipse IDE} with 
\href{https://projects.eclipse.org/projects/iot.embed-cdt}{Eclipse Embedded CDT}.
 
\section{uCsim Debug} \hypertarget{def:ucsim}{}
  
The uCsim debug console can be accessed with the telnet (1234 is the default port):
 \begin{verbatim}
 telnet localhost 1234
 \end{verbatim}
 
All \href{http://mazsola.iit.uni-miskolc.hu/\%7edrdani/embedded/ucsim/cmd.html}{uCsim commands} are supported.  

For windows users \href{https://www.putty.org/}{putty telnet client} is a good option to access the uCsim console. 
 
  
