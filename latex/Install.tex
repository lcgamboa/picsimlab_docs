\chapter{Install}

\section{Stable version executables download} 

If you are on Linux or Windows you can download the last version at:

\href{https://github.com/lcgamboa/picsimlab/releases}{https://github.com/lcgamboa/picsimlab/releases}

If you are on macOS you can run PICSimLab using Wine:

\begin{enumerate}
 \item  Download and install [`xquartz`](https://www.xquartz.org)
 \item  Download and install [Wine](https://dl.winehq.org/wine-builds/macosx/download.html)
 \item  Download the executable and double-click it to run the installer
\end{enumerate}

\section{Unstable version executables download}

The binaries of last code available on github can be downloaded at: \href{https://sourceforge.net/projects/picsim/files/latest\%20code\%20build\%20\%28unstable\%29/}{Sourceforge.net}
 
The unstable test version have the unreleased features of \href{https://github.com/lcgamboa/picsimlab/blob/master/CHANGELOG_auto.md}{Changelog\_auto.md}

If you need a specific binary that is not available please contact me. 

\section{Install from source}

\subsection{Linux}

 In Debian Linux and derivatives Linux native:

\textbf{Using a user with permission to run the sudo command:}

In first time build:
\begin{minted}[baselinestretch=1.2,fontsize=\footnotesize,bgcolor=colorbash]{bash}
git clone --depth=1 https://github.com/lcgamboa/picsimlab.git
cd picsimlab
./picsimlab_build_all_and_install.sh
\end{minted}

To recompile use:
\begin{minted}[baselinestretch=1.2,fontsize=\footnotesize,bgcolor=colorbash]{bash}
make -j4
\end{minted}

\subsection{Windows}

 Cross-compiling for Windows (from Linux or \href{https://docs.microsoft.com/windows/wsl/install-win10}{WSL} on win10)

In first time build in Debian Linux and derivatives target Windows 64 bits:

 \begin{minted}[baselinestretch=1.2,fontsize=\footnotesize,bgcolor=colorbash]{bash}
git clone https://github.com/lcgamboa/picsimlab.git
cd picsimlab
./picsimlab_build_w64.sh
\end{minted}

To recompile use:

\begin{minted}[baselinestretch=1.2,fontsize=\footnotesize,bgcolor=colorbash]{bash}
make FILE=Makefile.cross -j4 
\end{minted}

For target Windows 32 bits:

\begin{minted}[baselinestretch=1.2,fontsize=\footnotesize,bgcolor=colorbash]{bash}
git clone https://github.com/lcgamboa/picsimlab.git
cd picsimlab
./picsimlab_build_w32.sh
\end{minted}

To recompile use:
\begin{minted}[baselinestretch=1.2,fontsize=\footnotesize,bgcolor=colorbash]{bash}
make FILE=Makefile.cross_32 -j4 
\end{minted}

\subsection{macOS}

Theoretically it is possible to compile PICSimLab natively on macOS. But I do not have access to any computer
 with macOS to try to compile and until today nobody has communicated that they managed to do it. (help wanted) 


\subsection{Experimental version}
Experimental version

Experimental version can be built using the parameter "exp" on scripts:
\begin{minted}[baselinestretch=1.2,fontsize=\footnotesize,bgcolor=colorbash]{bash}
./picsimlab_build_all_and_install.sh exp
./picsimlab_build_w64.sh exp
./picsimlab_build_w32.sh exp
\end{minted}
And recompiled using the parameter "exp" on Makefiles:
\begin{minted}[baselinestretch=1.2,fontsize=\footnotesize,bgcolor=colorbash]{bash}
make -j4 exp
make FILE=Makefile.cross -j4  exp
make FILE=Makefile.cross_32 -j4 exp
\end{minted}
 
